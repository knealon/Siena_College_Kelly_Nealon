%%%%%%%%%%%%%%%%%%%%%%%%%%%%%%%%%%%%%%%%%%%%%%%%%%%%%%%%%%%%%%%%%%%%%%%%%%%%%%%%
% Introduction
%%%%%%%%%%%%%%%%%%%%%%%%%%%%%%%%%%%%%%%%%%%%%%%%%%%%%%%%%%%%%%%%%%%%%%%%%%%%%%%%

While particle physics captures the public's attention with huge detectors,
enormous accelerators, and outsized goals of understanding the universe
at its most basic level, modern experiments do not directly ``see" the particles.
To the non-physicist, these research efforts may seem at times to be
less substantial than astronomical research efforts that produce amazing
images of distant galaxies and nebulae. But there are devices that
can produce images of subatomic particles and these devices are relatively
easy and inexpensive to build.

When cosmic rays strike the upper atmosphere, they produce showers
of particles, most of which decay before they reach the surface
of the Earth. The decay products of these short-lived particles
include muons which do reach the surface of they earth. Early particle
physics experiments observed these particles by the tracks they left in
photographic emulsion plates.
Muons can also be observed using less expensive relative of a bubble chamber called
a {\it cloud chamber}. In this device, one cools a surface, usually using
dry ice, and allows isopropyl alcohol vapor to condense above
this cold surface. When fast-moving, subatomic particle pass
through this fog of alcohol, they ionize molecules along their travel path.
These ions provide condensation seeds where droplets of alcohol will form and
provide a visual record of where the particle has traveled.
These cloud chambers are quite inexpensive, requiring nothing more
than a clear container in which to constrain the alcohol vapor,
a light for easy viewing, and dry ice. Cosmic ray muons should produce
a track every few minutes or so, but one can acquire low-level radioactive samples
from online suppliers that will produce multiple ionizing particles
every second.

One challenge of running these cloud chambers is that the dry
ice necessary for their operation is not always readily available and
will eventually sublime away, limiting the length of time they can
be operated. To overcome this challenge, I will explore the use of
solid-state Peltier cooling devices\cite{peltier}, that
can create temperature gradients that necessary to create the alcohol vapor.
These cooling devices can be turned off and on with
the flick of a switch, removing the need for dry ice. A small company,
Nothing Labs\cite{nothinglabs}, has posted online
instructions on how to build one of these cloud chambers\cite{nothinglabs_cc}
and will even sell you a pre-built model for \$650.


